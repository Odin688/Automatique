\documentclass[12pt]{article}
%importartion des différents packages
\usepackage[utf8]{inputenc}  
\usepackage[T1]{fontenc}
\usepackage{geometry}
\usepackage{graphicx}
\usepackage{hyperref}
\usepackage{bm}
\usepackage{xcolor}
\geometry{top=15mm, bottom=20mm, left=20mm, right=20mm} % définit les marges
\renewcommand{\baselinestretch}{1.25}  % taille de l'interligne

\title{\bf \itshape Travaux Pratiques d'Automatique \\ Synthèse Générale}
\author{Basile Masson et Alexis Kittler}

\date{Année 2022-2023}

\begin{document}

\maketitle

\section{\itshape Introduction}

Tout au long de ces quatre séances de travaux pratiques d'Automatique, nous avons utilisé la maquette n°10 dans laquelle se trouvent les deux systèmes que nous allons identifier, corriger, et enfin asservir.

\section{\itshape Réponse harmonique}

\subsection{\itshape Travail préparatoire}

    \begin{itemize} % commande pour dire qu'on démarre une liste (à puces ici)

    \item \underline{Fonction de tranfert :} Rapport entre la sortie et l'entrée d'un système. Son module représente le rapport de l'amplitude/valeur efficace de la sortie sur celle de l'entrée et son argument représente le déphasage de la sortie par rapport à l'entrée.
    \item Avant de tracer le diagramme de Bode ou de Black, il faut déterminer la nature du système (passe-bas, passe-haut, coupe-bande, passe-bande) et sa bande-passante. Une fois ceci déterminé, on génère une entrée sinusoïdale avec le GBF et on se place au début de la bande passante en modifiant la fréquence du signal avec le GBF. \\ On affiche ensuite la sortie et l'entrée sur l'oscilloscope en simultané, et on mesure le module avec l'outil "Meas--> Amplitude" de l'oscillocope et le déphasage de la sortie par rapport à l'entrée "Meas --> Retard" à plusieurs fréquences parmi la bande passante.\\\\ En général, on mesure le plus de points quand la pulsation est à une décade de la pulsation de coupure. Il ne reste plus qu'à placer ces points, soit dans le diagramme de Bode, ou dans le diagramme de Black.
    \end{itemize}
\subsection{\itshape Travail expérimental}

\begin{itemize}
    \item \bf Système 1
\end{itemize}

\underline{Nature du système} : On prend f très grand, on remarque que la sortie est nulle. À l'inverse, pour f = 10 Hz, la sortie est amplifiée.


\end{document}

